\section{Criminal Justice Reform}
\label{sec:crim-just-reform}

Our criminal justice system is an embarrassment to our tradition and our values. The \#BlackLivesMatter movement is the most recent reminder of this obvious fact, and the most urgent demand for reform. That reform must happen over the course of the next presidential administration. I would commit to making that reform happen in the first two years of my administration.

But almost every problem that we see in the criminal justice system flows from the core inequality that we've allowed to grow within our society. The laws are harshest against the most politically disempowered (the poor); they are unenforced against the most politically empowered (Wall Street). And a bizarre and misguided policy of conservative Supreme Court justices to grant an expanding immunity to government officials who break the law has removed the only effective check on rogue police departments and federal agents.

As President, I would press for comprehensive reform of mandatory minimum sentences. We send people to prison for way too long, and when they get out, too many have no choice but to return to crime.

I would also direct the Justice Department to prosecute white collar crimes through penalties imposed on people first. Corporations are not people. When a corporation has been found to violate the law, we should hold its executives accountable. That includes, when the crime is serious enough or repeated, prison. And when fines are the only feasible penalty, they must be high enough to make it irrational for any corporation to risk their penalty. Too many corporations view fines as a cost of doing business. If they were a certain pathway to bankruptcy, fewer would violate the law.

I would also press Congress to ban practices that give governments an interest in finding their citizens guilty of crime. In Ferguson, for example, one of the most corrosive influences driving that crisis was a system that turned policemen into tax collectors, by making every infraction subject to a fine. Likewise with forfeiture statutes, which give the state a financial interest in crime. Any system of forfeiture should direct the assets taken to an entity unrelated to the government taking them.

I would also press for legislation that gives felons the choice to live a crime-free life again. I would press for legislation to end felon disenfranchisement, and provide support for ex-felons to transition into becoming productive members of society. These policies would cost money~---~but not as much as the costs of ex-felons returning to a life of crime.

Finally, I would press Congress to pass legislation that would reverse the Supreme Court's precedent that has granted government officials an increasingly large immunity when they violate the law. When police violate the law, the police department should pay. The federal government should establish streamlined alternative procedures for victims of lawless government officials to be compensated. But those who violate the law, including especially the rights protected by the Constitution, should face certain and swift consequences.

Except for this last reform, criminal justice reform now enjoys broad bipartisan support. I would leverage that support in the early days of my administration to push for comprehensive reform before the midterm elections. It is finally time for America to deliver on this most basic commitment of a civil society~---~that it treat its people decently and with due process, and that no one, especially government officials, be above the law.
