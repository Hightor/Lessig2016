\section{The Economy}
\label{sec:economy}

Americans are still suffering from the economic collapse triggered in part by Wall Street's recklessness. 
Though the economy has grown consistently since Congress passed Barack Obama's economic stimulus plan (described powerfully in Michael~Grunwald's The New New Deal), the growth is still historically anemic.

One of the reasons for this weakness has been the government's need to rely primarily on monetary stimulus, rather than fiscal stimulus. 
We have depended on the Fed to keep interest rates at historic lows as a way to spur economic activity, rather than the more traditional strategy of sustained public spending.

This was a mistake. 
Following the stimulus, America should have invested massively in infrastructure, to rebuild our crumbling bridges and roads, and to build-out an information superhighway to match for the digital age what Eisenhower did for automobiles. 
Had we done that, we would not only have created millions of jobs, we would have given our children the gift of a nation prepared for the 21st century.

But this mistake was forced onto our political system by unequal representation in the House of Representative. 
The ``no new taxes'' rule has an amplified role in American politics today because of the polarized nature of the House. 
That amplified polarization is created, in part, by the way Congress elects its members. 
The Citizen Equality Act would change that fundamentally, and let Congress follow the will of all the people, fairly represented.

As President, I would press for an investment in infrastructure of \$1.2 trillion over 5 years, financed through Rebuild America Savings Bonds.

I would also support legislation raising the minimum wage to \$15/hr.
