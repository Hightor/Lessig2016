\section{Protecting Our Environment}

Our values teach us that we don't own this planet. We hold it in trust for our children. And we must pass to them as rich and diverse a world as we inherited from our parents. This is a moral principle that it has taken too long for us to recognize, and act upon.

The most urgent environmental challenge we face~---~in light of this sacred trust~---~is climate change. The science is clear that climate change is happening~---~in part at least~---~because of human activity. In light of our values, that part alone should be enough to demand that we do something immediately to respond.

Yet it has been more than a decade since 2/3ds of America came to this view, and still, Congress has done nothing. Coal interests and oil interests have spent endless sums distorting the science and politicizing the policy, so that it is harder today to get legislation passed than it would have been a decade ago. This is one of the clearest examples of how money has corrupted our government. It is one of the biggest reasons why I took up the fight to end this corruption.

It is my view that polluters should clean up their pollution or pay the cost for the damage they've done. In this context, that principle translates into a strong demand that carbon emitters pay the price for what they do. I would support a carbon tax, which would force them to internalize the cost of what they're doing, thereby raising the price of dirty energy, and shifting demand to clean alternatives. I would support alternatives to a carbon tax only if I were convinced they could not be gamed to defeat the purpose of this fundamental moral principle.

That principle also means that we must do much more to stop the pollution of our air and waters by corporations who have convinced Congress to exempt them from this basic obligation. It saddens me to have to explain to my children why we can't eat too much of the fish we catch in Squam Lake, NH, because mercury from coal fired power plants has polluted the water and its wildlife.

Yes, clean energy would cost more. But whether we see the price or not, dirty energy costs much much more.

Of course, I also support the green energy policies proposed by the other Democratic candidates. But the rationale for those interventions would be less compelling if carbon polluters were paying the cost of their pollution. I would focus my administration's efforts on this fundamental fix first. For after it happens, there will be enormous incentives to innovate to find lower cost~---~and green~---~ways to support America's demand for energy.

But here too, the only way we will achieve this change is after The First Reform has passed. Even among Democrats, the power of oil and coal interests is strong. We have to free all representatives from this corrupting influence if they're to have the freedom to do the right thing. In my view, respecting the fundamental moral principle of the environment is the right thing.

