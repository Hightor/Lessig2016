\section{Tax Reform}
\label{sec:tax-reform}

Taxes in America should be progressive and efficient. 
Today they are neither~---~or at least, not enough. 
We have flattened the traditionally progressive income tax substantially. 
We have capped Social Security contributions at a regressively low amount. 
We tax corporations unequally and unfairly~---~and as economists have shown, regressively. 
And we have one of the most complex tax codes generally in the world.

But many of these problems tie directly to the corrupted system that we have in Con\-gress right now. 
The rich are the most important contributors to political campaigns. 
That produces an arms race among politicians to keep their taxes low. 
Corporations in America face a high nominal tax rate, but when all the exceptions and deductions are accounted for, it is one of the lowest rates in the world. 
Those exceptions and deductions are opportunities for congressmen to raise money. 
And the same with the complexity of taxes generally: every loophole is a fundraising opportunity. 
Our tax code is quickly becoming a device to raise money --- not for the United States Treasury, but for political campaigns.

I would work to radically simplify the tax code, while enhancing its progressivity. 
I would support abolishing the carried interest exemption that permits some of the richest Americans (hedge fund managers) to pay among the lowest tax rates. 
I would consider adding a higher marginal rate for individuals making more than \$1M a year. 
I would support abolishing the contribution cap for Social Security, applying the same Social Security rate to all levels of income. 
I would consider a proposal to offset any reduction in corporate tax with increased progressivity in income tax.

All of these changes follow directly from the principle of an efficient, progressive tax system. 
But they all will be vigorously opposed by the interests that dominate Washington right now. 
After we enact The First Reform, these changes will be easier to achieve.
