\section{The Emerging Surveillance Society}

The war on terror has led America to compromise a fundamental American value: that the state must leave us alone unless it has evidence to support invading our privacy. To protect this value, our Framers enacted the Fourth Amendment. Its central principle is that suspicionless searches are not the American way.

I would stop NSA surveillance on American citizens, wherever they may be. I would block any efforts by the NSA to outsource the violation of the Fourth Amendment to other governments or private contractors. And I would require that any surveillance system be built with both strict legal and technical protections to assure against misuse. Code must complement law, and no individual should have his or her privacy invaded unless the evidence justifying that invasion has been evaluated by a judge.

Millions of lobbying dollars will oppose this reform. The privatized intelligence industry in America has grown fat on federal contracts, and will only survive if America continues to spend billions to build the infrastructure of perfect surveillance. When we pass The First Reform, fighting these lobbyists will become easier. But long before that reform is passed, I would do everything I can as President to restore our government's commitment to the values embedded within our Constitution.

And I would export these values too. Technically, the Constitution does not protect non-Americans. But our values should. I understand the need to spy on terrorists and our enemies. But we should respect the privacy of people from wherever they come. As President, I would order an immediate review of the necessity and conditions under which we spy on non-citizens.

Finally, America owes a great debt to one of its own soldiers in this surveillance war, Edward Snowden. After concluding that every legal channel to challenge the illegal behavior that he knew our government was engaged in was closed to him, Snowden took an enormous personal risk to expose the crimes of our government. He did that with no hope of personal gain, and with the conscious recognition that the most likely consequence of his actions was that he would spend the rest of his life in prison.

It is not popular to state these obvious facts~---~just like it was not popular to defend Daniel Ellsberg forty years ago. But history will judge Snowden to be our generation's Ellsberg. He is a hero. And we need leaders with the courage to recognize that fact today, to at least say it, and then to act upon it.
