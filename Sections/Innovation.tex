\section{Innovation Policy}

Since the founding of America, the American government has secured to creators and inventors the protection of limited monopolies, to give them the the incentive they need to create.

I support this innovation policy. But I have long fought for reform in innovation policy, to fit the regulations of copyright and patent to the technology of the time.

As President, I would do the same. On copyright policy, I would convene Creative Rights Commission, composed of a broad range of disinterested but experienced individuals, who would survey the field of research and craft a blueprint for a copyright act for the 21st century. That act must achieve the primary objective of copyright law~---~to secure incentives to creators. But it must also reckon the public's interest in access to our culture and the spread and preservation of our cultural past.

I would also convene a similar Invention Commission, charged with reviewing the practice patent law, and its relation to innovation in the many fields in which it now operates. It too should recommend a blueprint for patent law in the 21st century, constrained not by the tradition it inherits, but only by the ultimate objective of patent and copyright law: the progress of science and the useful arts.

While those commissions do their work, I would oppose the current extremism in American copyright policy. I would veto any statute, and not support any trade agreement, that extended the terms of existing copyrights. I would defend the vigorous protection of fair use. I would seek broad exemptions from liability for both and private archivists. I would seek legislation that solved the orphan works problem in the most efficient way possible.

Finally, I would ask the Council of Economic Advisors to consider the creation of an Innovation Council. That body, independently funded with commissioners free of ties to industry, would be charged with evaluating the economic effect of copyright and patent policies, and whether those policies are advancing the primary aim of these regulations: to secure incentives to create and invent.

These policies too will not be possible until The First Reform is enacted. That was the insight that my friend Aaron Swartz convinced me of when he convinced me to give up my work in Internet and copyright policy to pursue fundamental reform of our government. But after that reform is passed, I am confident sanity in IP policy can return.
