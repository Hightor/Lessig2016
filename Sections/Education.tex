\section{Education}

Since the founding of this nation, we have recognized the need for the government to subsidize education. Education benefits the student. It benefits the society. We all gain from an educated people.

But over the last 20 years, we have lost sight of this fundamental truth. The consequence is a level of student debt that has created a national crisis. That debt now stands at \$1.2 trillion~---~more than auto loans (\$1 trillion), credit card debt (\$7oo million), or home equity loans (\$500 million). We need a workforce that is free to experiment and learn, not one that's terrified into dead-end jobs. No one should have to sell his soul to pay his debts.

The current crisis has been caused by a system that puts the weakest player (the student) in the middle of two very powerful forces (lenders and educators).

I don't believe the data show that subsidized student loans alone have caused education costs to rise. Much more significant has been the radical reduction in aid to education. But whatever the mix of causes, we must recognize an obvious point: As colleges increase costs, and the governments reduce educational subsidies, the weakest negotiator in this mix, the student, has been put in the middle. Faced with the choice between assuming endless debt or giving up a college education, most choose debt.

We need to avoid policies that remove the incentive of colleges to keep costs low. Our government needs to recommit to the principle at the core of American values since Jefferson~---~that society must subsidize education.

I would therefore support a substantial increase in federal subsidies to research and public educational institutions. I would press for policies that would forgive a portion of existing student debt, and permit students to refinance their debt at a low rate. I would support policies that condition federal subsidies for public education on efforts to restrain costs. And I would oppose policies that rely exclusively on students to create market pressure to keep educational costs in check.

I would also push Congress to support the policies of open access to scientific and educational material. There is no reason the American tax payer should have to pay for access to research that the federal government has funded. That research should be freely licensed. And as Open Education Resource activists have convincingly shown, we could radically decrease the cost of education material at both the primary and secondary level if the federal government supported the use of freely licensed textbooks. If we assigned just one open text book nationally, we could reduce the cost of textbooks by more than \mbox{\$2 billion.}

These policies would be blocked today by the powerful lobbies that support the existing federal subsidy to banks through loan guarantees, and the powerful private education market, whose students absorb a high proportion of student debt and have the highest default rates. The banks want complete deregulation~---~except for the government guarantees. Their potential profits translate into a very powerful lobby against reform.

The First Reform would weaken these influences, and give Congress an opportunity to address this complicated policy problem with the students' interest, and hence, our nation's interest, kept first.
