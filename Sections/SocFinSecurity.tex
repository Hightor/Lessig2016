
\section{Social \& Financial Security}
\label{sec:social--financial}

The market is a powerful engine for economic growth. 
We should encourage it where appropriate, and celebrate the achievements that are inspired by its freedom.

But there are spheres of social life that need to be protected from the market. 
Federal policy over the last 30 years has forgotten this obvious fact.

The aim of financial policy should be to secure a safe environment for finance to occur. 
We need to be able to trust that banks won't lose our money. 
Investors need confidence that massive speculation won't destroy the stability of the financial system.

For seventy years, the federal government assured this security, through regulations that kept the risk taking of large investors separated from the basic security of banks. 
That was achieved through many regulatory techniques, but the core was Glass-Steagall. 
When deregulation in 1990s undermined that separation, and Glass-Steagall was eventually repealed, this created the conditions that produced the collapse in 2008.

I support the return of Glass-Steagall, updated to modern conditions. 
It is critically important to Main Street that banks provide security first. 
That may be boring for bankers, but this part of the economy needs boring.

That change would change the economics of banking. 
I expect the move to consolidation would slow. 
But I would also consider the need to limit the size of banks to avoid ``too big to fail''.

The same principle of security explains the birth of Social Security. 
We as a society need to guarantee that when we are old, or if we are infirm, we will have the support we need. 
We secure that guarantee by setting these resources apart from the market. 
Privatizing Social Security is a sure path to insecurity for most.

Social Security has been weakened over the past 20 years. 
I would support increasing the benefits it pays. 
This is in part because private retirement resources have been decimated by the collapse of 2008 (and by the mistaken policy of moving away from defined benefit retirement plans).

I would also support removing the cap on contributions to Social Security (without necessarily proportionally increasing the benefits paid). 
Any increase in taxes will be incredibly difficult until The First Reform is enacted. 
But when Republicans and Democrats alike are free to consider the views of voters rather than funders, we will return to the objective of Social Security: security.
