
\section{National Security \& Foreign Policy}
\label{sec:national-security-}

Throughout our history, America's strength has come only in part from its might. Much more important are our actions for right.

Our sacrifice in World War II to save the world from fascism~---~and then, through the Marshall Plan, to help the defeated rebuild themselves~---~earned America the love of the free world. We didn't fight Hitler to get access to oil fields. We didn't enter the war in the Pacific because we really expected Japan to overrun us. Our grandparents made those sacrifices because it was the right thing to do. The free world understood that, and respected us for it.

We need to return to a foreign policy and a policy of national security that recognizes this truth.

After an horrific attack on American soil on 9/11, we reacted with understandable~---~and justified~---~anger.

But the war we fought in Iraq didn't make us any safer. The images of Abu Ghraib have recruited more terrorists against us than 100 Guantanamo Bays could hold. And the infrastructure of surveillance that we have helped build across the world has turned too many allies into critics. The world loves the American people and what we've created. But the world fears America. That is an astonishing reversal in an incredibly short time.

As America's Chief of State and Commander in Chief, I would press a foreign policy that shows the world who we actually are, and that celebrates the values that we, as a people, would defend. We need to work with allies as equals. We need to persuade the world to our view, not bully them. We need to be as great as we teach our children we have been, because we cannot survive in the modern world through force alone.

I am not a utopian. I am not a pacifist. I believe in military intervention to defend our people and for the cause of justice and humanity. I would not hesitate to use force when force is justified. And I accept that as the strongest economy in the world, we bear a disproportionate burden for defending the weak and our common interests in the world.

But we need to see that the world will never accept one nation over all others, and that a people unjustly treated will remember that injustice for many generations. When I was a student, during my days as a rabid anti-communist, I wandered through the Soviet Union and Eastern Europe. I met no one who showed hatred towards Americans, and I never feared for my safety as an American. But when my children are the same age, I fear there are too many places where they would know hatred towards us, and many places where they would be unsafe simply because of their passport. It will take time, but we have to change this reality. Our security as a nation, and peace in the world, depend on it.

As President, I would vigorously support the agreement with Iran. We must exercise a cautious skepticism of course, but we must also recognize that across history, such is always how peace begins. We have fundamental disagreements with Iran. We cannot tolerate the exporting of terrorism, and we must remain committed to defending the Israeli people. But this first step is an historic achievement. We should celebrate it, and work to make it succeed.

I recognize the frustration that many feel about Syria. After a decade, things are no better than they were. And given Russia's involvement, that will not change anytime soon. But we cannot ignore the incredible instability Syria is bringing to the region. Eleven million Syrians are displaced and effectively refugees. That fact will put enormous pressure on the region and Europe. And we cannot simply ignore that fact and hope it goes away.

I do not now support sending ground troops to Syria. But I do believe we must act with allies to secure at least humanitarian ends. I would support a policy like that which protected the Kurds. The topography of Syria will make that policy more difficult than it was in Kurdistan, but it could be effective nonetheless. There need to be safe zones in Syria. We need to work with allies to secure them now.

ISIS represents the most urgent challenge in the region. But I don't believe ISIS is like Al-Qaeda. I believe ISIS has engaged in horrific and criminal acts to recruit. But its ultimate objective is not to challenge us. Its objective is to achieve a dominant position in the region. Our strategy must be to work with the Sunnis and Iraq, to give the Sunnis a reason to support the Iraqi government against ISIS. I fear Iraq is looking elsewhere to build this anti-ISIS coalition. We need to move quickly to avoid an alliance that will only create more instability in the future.

I support the President's decision to maintain a small number of troops in Afghanistan. It may be that number should be higher than just 5,500. Those troops are not engaged in combat. They are only providing essential support. I fear that too hasty a retreat would bring to naught the sacrifices of so many Americans in that war. Obama made a commitment to get us out. But he is acting as a statesman, recognizing that rational policy doesn't necessarily track the life of an American administration.

Beyond the urgency of the Middle East, the most important long term foreign policy opportunity is China. For the first time, China is exercising its power internationally. I don't believe there is any longterm conflict between the interests of China and those of the United States. But I do believe that China will never accept a United States that doesn't treat it as an equal. We need to practice working out problems together. The progress the Obama administration has made with climate change policy is extraordinary. And we must work quickly to address the problems of cybersecurity.

The most difficult challenge we face in foreign policy is our evolving relationship with Russia. As a student of the transition in Russia, I am disappointed that the hopes for a true democracy, with powers that check each other, haven't yet been realized. But I believe that there is an enormous opportunity to build a relationship of understanding with Russia. We have different views and different values. But with allies across the world, I believe we can work with Russia for peace. That peace would be the most urgent of my foreign policy objectives.

Finally, we must recognize the instability that we induce by the very size and nature of our own defense budget. Eisenhower warned of the military-industrial complex. His original term was even better: the military-industrial-congressional complex. That complex creates enormous pressure for us to spend money on weapons of war. It puts enormous pressure on the United States government to encourage or at least tolerate the sale of weapons in contexts that could not induce peace. And it continues to put pressure on our government to spend money in ways that cannot help our long term security. The wars of tomorrow won't be fought in Bradley tanks, regardless of what the lobbyists say. Which is why, once again, we must pass The First Reform, and secure a Congress free to lead.